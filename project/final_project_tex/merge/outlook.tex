There are many things that can be improved with the toy shell-model code presented in this work. The latter is slow and only practically applicable on small model spaces. Further, at the moment it can only use the USDB interaction. This is a problem since different interactions could be necessary to include. The current code is tailored to only work with neutrons as valence nucleons, this makes it only possible to study a limited set of nuclei. When comparing theory to experiment it might be beneficial to study different observables, in the present condition our code can compute besides the eigenspectrum, total angular momentum and occupation numbers, but there are many other observables that could be of interest both to theoreticians and experimentalists.

That first issue that we would like to tackle if we are to continue this project is that of speed.
Currently our code uses an exact eigenvalue solver imported from numpy, which works fine for small models spaces. However, for larger model spaces iterative algorithms such as Lanczos is a better option. Some work on this has already been done, however we never finished it.

To further increase speed of shell-model codes, an interesting approach is to use the similarity between number-representation of Slater-determinants and binary numbers, and that most computers today uses binary numbers internally. We have been looking in to this and a skeleton of a shell-model code using this approach has been partially written in C++.

The next step of hypothetical improvements would be to allow for valence protons as well as the currently implemented valence neutrons. For the pure proton-proton part of such an improvement the same technology as for the already implemented neutron-neutron case can be used, so the only hard addition would be to implement the proton-neutron interactions.

Further improvements could involve implementation of spectroscopic factors. Since a lot of interesting phenomena in  nuclear physics involves nuclear decay this is a given improvement. Some attempts has been made already to do this, for the removal of neutrons. Unfortunately the authors underestimated the time it would take to make this work.

