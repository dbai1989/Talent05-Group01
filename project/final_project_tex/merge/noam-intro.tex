\section{Introduction of our Theoretical Framework}
The oxygen isotopes have been widely investigated \citep[for a review see][]{Brown2017}. To describe their low lying spectrum within the shell model, different model spaces and interactions are used, starting from an $^{16}$O core with extra neutrons in the $sd$-shell \cite{Kuo1966,Wildenthal1984,Brown1988,Brown2006}, adding the lower $p$-shell for core-excitation (intruder states) with particle-hole configurations \cite{Lawson1976} and incorporating the $pf$-shell as well \cite{Utsuno1999}.

In this work we investigate the structure of the $^{18-28}$O isotopes, both even and odd, by examining their low-lying states. This is done by developing our own shell model program and comparing its results to the NushellX@MSU program \cite{Brown2014}.

We use an $^{16}$O core, having 8 protons and 8 neutron at the $s$-$p$-shells, ($0s_{1/2},0p_{3/2},0p_{1/2}$). 
More neutrons are then excited in the $sd$-shell ($0d_{5/2},1s_{1/2},0d_{3/2}$), which serves as our model space. We use all possible configurations in these orbits and work in a harmonic oscillator basis with spin-orbit splitting. 
The Hamiltonian reads

		\begin{equation} \label{eq:H}
		\hat H = \hat H_0 + \hat H_I,
		\end{equation}
for
		\begin{equation} \label{eq:H_0}
		\hat H_0 = \sum_{p} \epsilon_p \hat n_{p},
		\end{equation}
the one-body Hamiltonian, where $\hat n_p = \hat a_p^\dagger \hat a_p$ is the number operator for the spherical orbit $p$ with quantum numbers ($n_p,\ell_p,j_p,m_p$) and $\epsilon_p = \braket{p|h_0|p}$ are the single-particle energies (SPE). The interaction part reads
		\begin{equation} \label{eq:V}
		\hat H_I = \sum_{p < q=1}^{N} \sum_{r < s=1}^{N} V(p,q;r,s) \hat T(p,q;r,s),
		\end{equation}
		with
		\begin{equation} \label{eq:P}
		\hat T(p,q;r,s)  = \hat a_p^\dagger \hat a_q^\dagger \hat a_r \hat a_s,
		\end{equation}

$\hat H_I$ is given in $M$-scheme and is the two-body density operator for nucleon pairs in orbits $p,q$ and $r,s$ coupled to the total spin projection $M$, where $N$ is the number of particles in the configuration. In $J$-scheme $\hat H_I$ reads 
\begin{equation}
	\hat H_I =  \sum_{a\leq b,c \leq d} \sum_{JT} V_{J,T}(p,q;r,s) \hat T_{J,T}(p,q;r,s),
\end{equation}
where $\hat T_{J,T}(p,q;r,s)$ is the scalar two-body density operator for nucleon pairs in orbits $p,q$ and $r,s$ coupled to spin quantum numbers $J,M$ and isospin quantum numbers $T,T_z$ \cite{Brown2006}. Here the appropriate quantum numbers are $(n_i,\ell_i,j_i)$, $i \in \{a,b,c,d\}$. The transformation between the two-body matrix elements (TBME) from $J$- to $M$-scheme reads
\begin{multline}
	V(p,q;r,s)  =  \braket{j_{p},m_{p};j_{q},m_{q}|V|j_{r},m_{r};j_{s},m_{s}} \\ 
			    =  \mathcal{N}_{pq} \mathcal{N}_{rs} \sum_{J} \braket{j_{p},m_{p};j_{q},m_{q}|JM} \braket{j_{r},m_{r};j_{s},m_{s}|JM} \\
			   	\times \braket{(j_{p},j_{q})J||V||(j_{r},j_{s})J} \delta_{m_p+m_q,m_r+m_s},
\end{multline}
where other quantum numbers are implicitly implied. Here we have used the Wigner-Eckart theorem with the fact that $ V $ is a rank-zero tensor. Furthermore, $\braket{j_{a},m_{a};j_{b},m_{b}|JM}$ is a Clebsch-Gordan coefficient, $ \mathcal{N}_{pq} $ ($ \mathcal{N}_{rs} $) is a normalization factor and is equal $ \sqrt{2} $ if $ p=q $ ($ r=s $) with only even values of $ J $ and 1 if $ p \not=q $ ($ r \not=s $).

We use the SPE and TBME of the USDB interaction \cite{Brown2006} and work in $M$-scheme. The SPE values and order are given in Table \ref{tab:SPE}. The TBME for $A=18$ are given in \cite{Brown2006} in $J$-scheme for $T=1,0$ in Tables I and II, respectively. As was done for the USD interaction \cite{Wildenthal1984}, the SPE are taken to be mass independent and for the TBME we employ a mass dependence of the form
\begin{equation}
	V(p,q;r,s)(A) = \left( \frac{18}{A} \right)^p V(p,q;r,s)(A=18),
\end{equation}
with $p=0.3$. This qualitative mass dependence is expected from the evaluation of a medium-range interaction with harmonic-oscillator radial wave functions. It also defines TBME for other $A$ values in the $sd$-shell.

Using the single-particle states (SPS) we construct the appropriate slater determinants according to the number of particles which we place in the $sd$-shell. This enables us to construct expectation values of the Hamiltonian \eqref{eq:H} and diagonalize it to obtain the energies. 

In this work we represent a new shell-model code, compare it with NushellX results and conduct further investigation of the oxygen isotopes' wave functions using NushellX.

\begin{table}[H]
\caption{Single particle energies of the $sd$-shell in the $M$-scheme basis with their corresponding quantum numbers: $N$, the principle quantum number; $\ell$, the orbital angular momentum; $J$, the total angular momentum; $M_j$, the total angular momentum projection on the $z$ axis. \label{tab:SPE}}
\begin{ruledtabular}
\begin{tabular}{c|cccccc}
index	&	$N$	&	$\ell$	&	$J$	&	$M_j$	&	SPE			\\
\hline 
1		&	1	&	0		&	1	&	$-1/2$	&	-3.20790	\\
2		&	1	&	0		&	1	&	$+1/2$	&	-3.20790	\\
3		& 	0	&	2		&	3	&	$-3/2$	&	 2.11170	\\
4		&	0	&	2		&	3	&	$-1/2$	&	 2.11170	\\
5		&	0	&	2		&	3	&	$+1/2$	&	 2.11170	\\
6		&	0	&	2		&	3	&	$+3/2$	&	 2.11170	\\
7		&	0	&	2		&	5	&	$-5/2$	&	-3.92570	\\
8		&	0	&	2		&	5	&	$-3/2$	&	-3.92570	\\
9		&	0	&	2		&	5	&	$-1/2$	&	-3.92570	\\
10		&	0	&	2		&	5	&	$+1/2$	&	-3.92570	\\
11		&	0	&	2		&	5	&	$+3/2$	&	-3.92570	\\
12		&	0	&	2		&	5	&	$+5/2$	&	-3.92570	
\end{tabular}
\end{ruledtabular}
\end{table}