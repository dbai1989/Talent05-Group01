\section{LNLT\_shell} \label{sec:CodeExpl}
% General Introduction about the shell model code
%% Toy code, with goal to compute oxygen isotope spectrum in the sd shell

Throughout this project a simple shell-model code, called ``LNLT\_shell'', has been developed in python. The main goal of this shell-model is to be able reproduce the oxygen isotopes spectra in the sd-shell.

A nuclear shell-model code approximate the spectrum for an atomic Nucleus with \(A\) nucleons by assuming a frozen core with \(A_{\rm{core}}\) nucleons, and then solves the many-body Schrödinger equation \(H\ket{\Psi}=E\ket{\Psi}\) where \(H\) is described in equation \ref{eq:H}
for a group of \(n=A-A_{\rm{core}}\) valence nucleons.


% The general structure
%% Parse arguments (more about what arguments that exist rather that how)

The program starts by parsing the argument list. There are six different arguments:
\begin{itemize}
\item[-n n:] Used to specify number of valence nucleons.
\item[-M M:] Used for total-M restriction, discussed below.
\item[-of filename:] To specify what json file contains the shells.
\item[-os bool:] To turn the pairing basis on and off.
\item[-if filename:] To specify what file contains the two-body matrix elements.
\item[-o filename:] To specify what file to write the output to.
\end{itemize}
observe that -of and -if are mutually exclusive.


%% Reading/setting up the shells and single particle basis


The first thing the code does after parsing the arguments is setting up the single particle basis.
This is done either by reading them directly from the m-scheme matrix element file %\cite Morten and Alex
or by constructing them from shells it reads in a given json file.

% Format n l 2j 2m ?

%% Setting up all possible Slater Determinants

When the single particle basis is determined, all possible Slater determinants for the valence nucleons are constructed to form the many-body basis.
This is done by constructing all unique combinations of \(n\) single particle states, each such combination representing a Slater determinant.
%% M-Restriction
After the many-body basis is constructed, it is possible to apply total M-restriction to exploit that the used Hamiltonian is a spherical scalar and therefore block-diagonal in total \(M\). This is done by removing all Slater determinants that do not have the desired total \(M\).

%% Discuss the TwoBodyOperator class, not go in to detail but the general ideas
The many-body Hamiltonian matrix is constructed in the class TwoBodyOperator, by acting with \(\bra{a,b}V\ket{c,d}_{AS}a^\dagger_a a^\dagger_b a_d a_c\) repeatedly for different values of \(a,\,b,\,c,\,d\) on all the Slater determinants in the many-body basis to determine which states differ with up to two single-particles states. This process results in a full matrix
%% The diagonalisation using numpy
which then is diagonalized by applying a full diagonalization routine from numpy. % cite numpy?
This results in list of eigen energies \(E\) with associated eigen states \(\ket{E,i}\) where \(i\) takes in to account potential degeneracy.

%% The J^2 operator
When the eigenstates to the Hamiltonian has been computed, the total angular momentum of each state is computed by solving  \(\hbar^2 J(J+1) = \bra{E,i}\hat{J}^2\ket{E,i}\) for each state. The operator \(\hat{J}^2\) can be divided in a two- and one-body operator, from which a many-body matrix can be obtained with the same machinery described for the full Hamiltonian matrix.

%% Occupation numbers
For each state shell-occupation number is determined by the application of
\begin{equation}
  \bra{E,i}n_k\ket{E,i}=\bra{E,i}\sum_{m=-j}^j a^\dagger_{k,m}a_{k,m}\ket{E,i}
\end{equation}
where \(k\) represents the shell, and \(j\) is associated to shell \(k\).

%% The output format
The program writes the results to the screen and optionally to a user specified file, where the single-particle basis, the many-body basis and the eigen-spectrum of the Hamiltonian.

Different interactions can be provided in two different ways. Either by a file which format will be described below, or as a class containing a method get\_matrix\_element that takes four single particle states as its arguments.
The matrix element file starts by stating how many single-particle states there are, and then continues listing them together with the single-particle energies. There after the file contains a number \(l\) representing how many two-body matrix elements there exists, followed by \(l\) lines of four integers representing the two-body states on the bra and ket side of the two-body matrix element and one fixed point value with \(5\) decimals and one hole number.
