%\documentclass[aps,prl,groupedaddress]{revtex4-1}  % One column - tight
%\documentclass[aps,prl,preprint,groupedaddress]{revtex4-1}  % One column - spread
\documentclass[aps,prl,reprint,groupedaddress]{revtex4-1}  % Two column
%\documentclass[aps,prl,preprint,superscriptaddress]{revtex4-1}
%\documentclass[aps,prl,reprint,groupedaddress]{revtex4-1}

% Use the \preprint command to place your local institutional report
% number in the upper righthand corner of the title page in preprint mode.
% Multiple \preprint commands are allowed.
% Use the 'preprintnumbers' class option to override journal defaults
% to display numbers if necessary
%\preprint{}

% You should use BibTeX and apsrev.bst for references
% Choosing a journal automatically selects the correct APS
% BibTeX style file (bst file), so only uncomment the line
% below if necessary.
%\bibliographystyle{apsrev4-1}

% % % % % % % % % % % % Noam% % % % % % % % % % % % 
% Added preamble commands:
\usepackage[utf8]{inputenc}
\usepackage{array}
\usepackage{mathrsfs}
\usepackage{multirow}
\usepackage{amsmath}
\usepackage{graphicx}
\graphicspath{{figures/}{figures/phase_diagram/}{figures/levels_scheme/}{figures/decomposition/}{figures/phase_diagram/}{figures/energies_transitions_ratios/}}
\usepackage[unicode=true,pdfusetitle,bookmarks=true,bookmarksnumbered=false,bookmarksopen=false,
 breaklinks=false,pdfborder={0 0 1},backref=false,colorlinks=false]{hyperref}
%%%%%%%%%%%%%%%%%%%%%%%%%%%%%% User specified LaTeX commands.
\usepackage{braket}
\hypersetup{colorlinks=True,urlcolor=blue,linkcolor=blue,citecolor=blue,filecolor=black}
\usepackage[caption=false]{subfig}
% Extra commands
%\usepackage{lmodern}
%\usepackage[T1]{fontenc}
%\setcounter{secnumdepth}{3}
%\usepackage{color}
%\usepackage{float}
%\usepackage{mathtools}
%\usepackage{amssymb}
%\usepackage{breakurl}

%\usepackage{pslatex}  % Change font style.
%\renewcommand{\arraystretch}{1.5}  % Widens rows in Tables
%\setlength{\extrarowheight}{2pt}   % Widens rows in Tables

\makeatletter

%%%%%%%%%%%%%%%%%%%%%%%%%%%%%% LyX specific LaTeX commands.
%% Because html converters don't know tabularnewline
\providecommand{\tabularnewline}{\\}
%% A simple dot to overcome graphicx limitations
\newcommand{\lyxdot}{.}

\makeatother

\begin{document}

%Title of paper
\title{Introduction of our theoretical framework}

% repeat the \author .. \affiliation  etc. as needed
% \email, \thanks, \homepage, \altaffiliation all apply to the current
% author. Explanatory text should go in the []'s, actual e-mail
% address or url should go in the {}'s for \email and \homepage.
% Please use the appropriate macro foreach each type of information

% \affiliation command applies to all authors since the last
% \affiliation command. The \affiliation command should follow the
% other information
% \affiliation can be followed by \email, \homepage, \thanks as well.
\author{N. Gavrielov}
%\email[]{Your e-mail address}
%\homepage[]{Your web page}
%\thanks{}
%\altaffiliation{}
\affiliation{}

%Collaboration name if desired (requires use of superscriptaddress
%option in \documentclass). \noaffiliation is required (may also be
%used with the \author command).
%\collaboration can be followed by \email, \homepage, \thanks as well.
%\collaboration{}
%\noaffiliation

\date{\today}

\begin{abstract}
% insert abstract here
\end{abstract}

% insert suggested PACS numbers in braces on next line
\pacs{}
% insert suggested keywords - APS authors don't need to do this
%\keywords{}

%\maketitle must follow title, authors, abstract, \pacs, and \keywords
\maketitle


\section{Introduction}

We investigate the structural changes of the $^{18-28}$O isotopes, both even and odd isotopes, by examining the energies of their low-lying states. This is done by developing our own shell model program and comparing the energies it generates to the NushellX@MSU program \cite{Brown2014}. 
We use a frozen $^{16}$O core, having 8 protons and 8 neutron at the $sp$-shell ($0s_{1/2},0p_{3/2},0p_{1/2}$). 
More neutrons are then excited in the $sd$-shell ($0d_{5/2},1s_{1/2},0d_{3/2}$), which serves as our model space. We use all possible configurations in these orbits and work in a harmonic oscillator basis with spin-orbit splitting. 
The model space Hamiltonian reads

		\begin{equation} \label{eq:H}
		\hat H = \hat H_0 + \hat H_I,
		\end{equation}
		where $\hat H_0$ is the unperturbed one-body Hamiltonian
		\begin{equation} \label{eq:H_0}
		\hat H_0 = \sum_{p,q} \braket{p|h_0|q} \hat a_{p}^\dagger \hat a_{q},
		\end{equation}
		and $\hat H_I$ is the pair breaking interaction
		\begin{equation} \label{eq:V}
		\hat H_I = \sum_{p,q,r,s} \braket{p,q|V|r,s}\hat P_{p,q}^+ \hat P_{r,s}^- ,
		\end{equation}
		with
		\begin{equation} \label{eq:P}
		\hat P_{p,q}^+ = \sum_{p,q} \hat a_p^\dagger \hat a_q^\dagger, \qquad
		\hat P_{r,s}^- = \sum_{r,s} \hat a_r \hat a_s,
		\end{equation} 
where $p,q,r,s$  represent all possible single-particle states (SPS) quantum numbers. The Hamiltonian is rotationally invariant 

For the two-body matrix elements (TBME) \textbf{(define them as in \cite{Brown2006})} in Eq. \eqref{eq:V} we use the USDB interaction TBME \cite{Brown2006} (Tables I and II in \cite{Brown2006}, given in $J$-scheme, for $T=1,0$ respectively), and the single-particle energies (SPE) ($1s_{1/2},0d_{3/2},0d_{5/2}$) are ($-3.2079, 2.1117, -3.9257$). 
We work in $M$-scheme, where the SPS are ordered as given in Table \ref{tab:SPE}. 

\textbf{The transformation from $J$-scheme reads?}

Using the SPS we construct the appropriate slater determinants, $\ket{\psi}$, according to the number of particles (even or odd)  which we place in the $sd$-shell.

\begin{table}[h]
\caption{Single particle energies of the $sd$-shell in the $M$-scheme basis with their corresponding quantum numbers: $N$, the principle quantum number; $\ell$, the orbital angular momentum; $J$, the total angular momentum; $M_j$, the total angular momentum projection to the $z$ axis.. \label{tab:SPE}}
\begin{ruledtabular}
\begin{tabular}{c|cccccc}
index	&	$N$	&	$\ell$	&	$J$	&	$M_j$	&	SPE			\\
\hline 
1		&	1	&	0		&	1	&	$-1/2$	&	-3.20790	\\
2		&	1	&	0		&	1	&	$+1/2$	&	-3.20790	\\
3		& 	0	&	2		&	3	&	$-3/2$	&	 2.11170	\\
4		&	0	&	2		&	3	&	$-1/2$	&	 2.11170	\\
5		&	0	&	2		&	3	&	$+1/2$	&	 2.11170	\\
6		&	0	&	2		&	3	&	$+3/2$	&	 2.11170	\\
7		&	0	&	2		&	5	&	$-5/2$	&	-3.92570	\\
8		&	0	&	2		&	5	&	$-3/2$	&	-3.92570	\\
9		&	0	&	2		&	5	&	$-1/2$	&	-3.92570	\\
10		&	0	&	2		&	5	&	$+1/2$	&	-3.92570	\\
11		&	0	&	2		&	5	&	$+3/2$	&	-3.92570	\\
12		&	0	&	2		&	5	&	$+5/2$	&	-3.92570	
\end{tabular}
\end{ruledtabular}
\end{table}

\appendix

\section{My TOC}

\subsection{Background}
\begin{enumerate}
	\item Description for the Oxygen isotopes - the physical phenomenon we investigate.
	\item The $1s0d$-shell model space.
	\item Pairing Hamiltonian and pair breaking Hamiltonian (We use second quantization. Is this only in m-scheme?).
	
		\begin{equation}
		\hat H = \hat H_0 + \hat V
		\end{equation}
				
		\begin{equation}
		\hat H_0 = \xi \sum_{p,\sigma} (p-1) \hat a_{p\sigma}^\dagger \hat a_{p\sigma},
		\end{equation}
		
		\begin{equation}
		\hat V = \sum_{p \leq q} \braket{p|V|q}\hat P_p^+ \hat P_q^- ,
		\end{equation}
		
		\begin{equation}
		\hat P_p^+ = \sum_{\sigma_p, \geq 0} \hat a_{p, \sigma_p}^\dagger \hat a_{p, -\sigma_p}^\dagger, \qquad
		\hat P_p^- = \sum_{\sigma_p, \geq 0} \hat a_{p, -\sigma_p} \hat a_{p, \sigma_p}.
		\end{equation}
		
		Aren't we using
		\begin{equation}
		\hat V = \sum_{p \leq q} \braket{pq|V|rs}\hat P_{p,q}^+ \hat P_{r,s}^- ?
		\end{equation}
		
		\begin{equation}
		\hat P_{p,q}^+ = \sum_{\sigma_p,\sigma_q, \geq 0} \hat a_{p, \sigma_p}^\dagger \hat a_{q, -\sigma_q}^\dagger, \qquad
		\hat P_{r,s}^- = \sum_{\sigma_r,\sigma_s, \geq 0} \hat a_{r, -\sigma_r} \hat a_{s, \sigma_s}??
		\end{equation}
\end{enumerate}


\subsection{Introduction of our theoretical framework}
\begin{enumerate}
	\item The $sd$-shell serves as our model space.
	\item The Hamiltonian we are using.
	\begin{enumerate}
		\item Few words on the pair breaking interaction.
		\item The Hamiltonian is rotationally invariant.
	\end{enumerate}
	\item SPS coupling to $M_{tot}$. For $M_{tot}=0$ ($M_{tot}=1/2$)  we obtain all possible $J$ values for even (odd) nuclei. Explain this when comparing our results to NushellX.
	\item Matrix elements in m-scheme. 
	\begin{enumerate}
		\item Translate Eq. (19) in \cite{Brown2006} to M-scheme? 
		\item The USD Hamiltonian is defined by 63 sd-shell TBME and three SPE given in Table I of [1]. This is converted to the $M$-scheme and was given to us by Morten in the sdshellint.dat file. To this we added the refinement of Eq. (19) in \cite{Brown2006}.
	\end{enumerate}
	\item Slater determinants.
	\item We solve the many-body Schrodinger eq.
	\item The different NushellX interactions:
	\begin{enumerate}
		\item USD (the original, by Winldenthal)
		\item USDB.
		\item USDA.
		\item CCEI (Gustav's).
	\end{enumerate}
\end{enumerate}


\bibliography{/home/noam/Desktop/Physics/Articles/library.bib}
\end{document}


